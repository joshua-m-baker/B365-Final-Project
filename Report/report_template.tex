%%%%%%%%%%%%%%%%%%%%%%%%%%%%%%%%%%%%%%%%%
% Stylish Article
% LaTeX Template
% Version 2.1 (1/10/15)
%
% This template has been downloaded from:
% http://www.LaTeXTemplates.com
%
% Original author:
% Mathias Legrand (legrand.mathias@gmail.com) 
% With extensive modifications by:
% Vel (vel@latextemplates.com)
%
% License:
% CC BY-NC-SA 3.0 (http://creativecommons.org/licenses/by-nc-sa/3.0/)
%
%%%%%%%%%%%%%%%%%%%%%%%%%%%%%%%%%%%%%%%%%

%----------------------------------------------------------------------------------------
%	PACKAGES AND OTHER DOCUMENT CONFIGURATIONS
%----------------------------------------------------------------------------------------

\documentclass[fleqn,10pt]{SelfArx} % Document font size and equations flushed left

\usepackage[english]{babel} % Specify a different language here - english by default

\usepackage{lipsum} % Required to insert dummy text. To be removed otherwise

%----------------------------------------------------------------------------------------
%	COLUMNS
%----------------------------------------------------------------------------------------

\setlength{\columnsep}{0.55cm} % Distance between the two columns of text
\setlength{\fboxrule}{0.75pt} % Width of the border around the abstract

%----------------------------------------------------------------------------------------
%	COLORS
%----------------------------------------------------------------------------------------

\definecolor{color1}{RGB}{0,0,90} % Color of the article title and sections
\definecolor{color2}{RGB}{0,20,20} % Color of the boxes behind the abstract and headings

%----------------------------------------------------------------------------------------
%	HYPERLINKS
%----------------------------------------------------------------------------------------

\usepackage{hyperref} % Required for hyperlinks
\hypersetup{hidelinks,colorlinks,breaklinks=true,urlcolor=color2,citecolor=color1,linkcolor=color1,bookmarksopen=false,pdftitle={Title},pdfauthor={Author}}

%----------------------------------------------------------------------------------------
%	ARTICLE INFORMATION
%----------------------------------------------------------------------------------------

\JournalInfo{Introduction to Data Analysis  and Mining 2018} % Journal information
\Archive{} % Additional notes (e.g. copyright, DOI, review/research article)

\PaperTitle{Semester Project} % Article title

\Authors{William Huibregtse\textsuperscript{1}, Joshua Bakere\textsuperscript{1}, Chris East\textsuperscript{1}} % Authors
\affiliation{\textsuperscript{1}\textit{Computer Science, School of Informatics , Computing and Engineering, Indiana University, Bloomington, IN, USA}} % Author affiliation


\Keywords{Keyword1 --- Synergy --- Keyword3} % Keywords - if you don't want any simply remove all the text between the curly brackets
\newcommand{\keywordname}{Keywords} % Defines the keywords heading name

%----------------------------------------------------------------------------------------
%	ABSTRACT
%----------------------------------------------------------------------------------------

\Abstract{Include abstract here -- A summary of your work}

%----------------------------------------------------------------------------------------

\begin{document}

\flushbottom % Makes all text pages the same height

\maketitle % Print the title and abstract box

\tableofcontents % Print the contents section

\thispagestyle{empty} % Removes page numbering from the first page




%----------------------------------------------------------------------------------------
%Problem and Data Description
%----------------------------------------------------------------------------------------


\section{Problem and Data Description} % The \section*{} command stops section numbering




First we want to get a general idea of our data set and get a deeper understanding of the underlying structure.\\
There are 59 named features or variables for our data set. \\
With 892816 observations for training and 595212 for test\\
There are no duplicate observations. \\
Features that belong to similar groupings are given certain feature names.\\
- Ind: related to individual or driver\\
- Reg: related to geographical region\\
- Car: related to car being insured\\
- Calc: are calculated features done by Proto themselves\\
Postfix descriptors describes the features data type.\\
- Bin: Binary (1 or 0)\\
- Cat: Categorical *Note: the dataset has the categorical data already convert into factors and then integers\\
- All other variables are either integer or numeric\\
As stated the Data Types are numeric and integer, with integer being the predominant type 49 to 10.\\
Missing values are represented by -1.\\
In total, there are 13 variables with missing values.\\
There is Target feature which denotes the binary classification for that observation. This feature is the feature we are trying to learn/predict for the test data.\\
There is an ID feature which is an anonymized identities of insured drivers.  \\

Porto Seguro’s Safe Driver Prediction has 59 variables and 1.3 million observations, which qualifies as a good candidate for reducing overall dimensions of the data to significantly increase the speed of analysis techniques at the cost of more upfront data processing.


There are only 21694 cases of classification 1, which is 3.64 percent of the observations in the training data set, showing significant skew in the expected class towards a ``0'' prediction.


\bigskip
\bigskip

%----------------------------------------------------------------------------------------
%	Data Preprocessing $\&$ Exploratory Data Analysis
%----------------------------------------------------------------------------------------

\section{Data Preprocessing $\&$ Exploratory Data Analysis} % The \section*{} command stops section numbering

\subsection{Feature Engineering}

As even missing data can be significant, a new feature was added to the data set. This feature was the count of missing values for each entry before these missing values were processed. This technique allows the retention of the potentially useful information provided by the missing values.
\subsection{Handling Missing Values}

After observing the summaries of each variable in our data sets, it was clear that variables ps-car-03-cat and ps-car-05-cat contained mostly missing values for each data set. Because we later used a missing value replacement techique to process the data, applying this techique to variables with mostly missing values may have overfitted and affected the outcome of future analysis, thus both because of this possible overfitting and that useful information may be captured in the engineered feature of missing value counts, these columns were removed before proceeding with our NA replacement technique.

After the columns with mostly missing were removed, we replaced missing values with the mean of it's variable using the R package ``mice''.
\subsection{Exploratory Data Analysis}

Add subsections if needed.
\bigskip
\bigskip
%----------------------------------------------------------------------------------------
 % Algorithm and Methodology
%----------------------------------------------------------------------------------------


\section{Algorithm and Methodology}

Briefly explain the algorithms in this section, i.e., linear regressions. You can add more subsections if needed, i.e., regression trees etc.


\bigskip
\bigskip
%----------------------------------------------------------------------------------------
 % Experiments and Results
%----------------------------------------------------------------------------------------
\section{Experiments and Results}


\bigskip
\bigskip
%----------------------------------------------------------------------------------------
 % Summary and Conclusions
%----------------------------------------------------------------------------------------
\section{Summary and Conclusions}
\bigskip
\bigskip
\bigskip



\phantomsection
\section*{Acknowledgments} % The \section*{} command stops section numbering

\addcontentsline{toc}{section}{Acknowledgments} % Adds this section to the table of contents



%----------------------------------------------------------------------------------------
%	REFERENCE LIST
%----------------------------------------------------------------------------------------
\phantomsection
\bibliographystyle{unsrt}
\bibliography{sample}

%----------------------------------------------------------------------------------------

\end{document}